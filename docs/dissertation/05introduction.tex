\chapter{Introduction}

This dissertation explores the use of Prolog in the browser by building a Prolog interpreter, WebPL, that runs in the browser using WebAssembly. This chapter introduces the motivation for the project, briefly discussing the existing landscape of Prolog implementations for the browser, and outlines the aims of the project.

\section{Motivation}

\label{sec:motivation}

Web pioneer Marc Andreessen claimed in the early days of the Web that ``the browser will be the operating system'': the browser would become the platform on which modern applications are built \cite{kosnerAlwaysEarlyMarc2012}. With increasingly sophisticated applications being delivered through the browser, this vision is proving more and more correct. In the last decade, the development of WebAssembly, a low-level binary instruction format for the Web, has further bridged the gap between browser-based and native applications, enabling web applications to run with near-native performance \cite{haasBringingwebspeed2017}.

The Prolog logic programming language is not exempt from this trend. Prolog is used on the Web in a variety of applications, including for browser-based development environments like SWISH \cite{wielemakerSWISHSWIPrologSharing2015}, enforcing dependency constraints in JavaScript packages\footnote{\url{https://v3.yarnpkg.com/features/constraints}}, and as a query language for databases \cite{wielemakerUsingPrologFundament2007}.

Traditional Prolog web applications typically use a client-server architecture. The Prolog development environment SWISH is one such application: the SWI-Prolog engine runs on the server, communicating with the browser using Pengines, a library for exchanging Prolog queries and results over HTTP \cite{lagerPenginesWebLogic2014}. There are several drawbacks of this model. A dedicated server is required to run the Prolog engine, which may be costly, and latency is introduced by the need for network communication. Running the Prolog engine in the browser directly would eliminate these drawbacks; indeed, Flach concludes that ``an in-browser solution is probably the future of Prolog on the web'' \cite{flachSimplyLogicalFirst2023}.

There are several Prolog implementations that run in the browser, which fall into two categories: general-purpose native implementations that have been ported to WebAssembly, which I will call \emph{web-ported} implementations, and implementations that have been designed from the ground up specifically for the Web, which I will call \emph{web-native} implementations. The former category includes SWI-Prolog \cite{wielemakerSWIProlog2012} and Trealla Prolog \cite{davisonTreallaProloghttps2020}, while a notable example of the latter is Tau Prolog \cite{riazaTauPrologProlog2024}, written in JavaScript. At the time of writing, there are no web-native implementations that use WebAssembly.

Web-ported implementations enjoy WebAssembly's near-native performance but suffer from poor integration with the browser environment and large binary sizes. The resource requirements of applications in the browser differ greatly from those of native applications, particularly in terms of memory, and web-ported implementations are typically heavyweight and do not take this into account.

Web-native implementations, on the other hand, are more tightly integrated with the browser environment, but do not yet have the performance benefits of WebAssembly. Furthermore, JavaScript's memory management is entirely independent of the Prolog engine, so information about the state of the Prolog engine cannot be used to optimise allocation and deallocation of memory.

I hypothesise that \textbf{a web-native Prolog implementation using WebAssembly, with design guided by the constraints of the browser environment, may be able to achieve superior performance compared to existing implementations, while preserving tight integration with the browser}. The project explores this hypothesis by building such a Prolog implementation.

\section{Project Aims and Outcomes}

The aims of the project were as follows:

\begin{itemize}
\item to \textbf{build a web-native pure Prolog\footnote{A subset of Prolog that does not use cut, negation, or extra-logical predicates.} implementation} that uses WebAssembly, with a particular focus on performance and browser integration, including
\begin{itemize}
\item a \textbf{lexer} and \textbf{parser}, and
\item an \textbf{interpreter} for the resulting abstract syntax tree,
\end{itemize}
\item to \textbf{explore optimisations} to improve performance, including both traditional Prolog optimisations and those that specifically target the browser environment, and
\item to \textbf{evaluate its performance}, both in terms of execution time and memory usage, against existing web-ported and web-native Prolog implementations, exploring which factors contribute to any performance differences.
\end{itemize}

The following extensions were implemented:

\begin{itemize}
\item to \textbf{extend the implementation} to support more advanced Prolog features, such as extra-logical predicates, cut, and negation,
\item to \textbf{build a browser-based development environment} for the Prolog implementation, similar to SWISH, which supports \textbf{plugins} for other Prolog implementations to facilitate comparison,
\item to \textbf{implement a garbage collector} to make more efficient use of memory, and
\item to \textbf{develop a foreign function interface}, extending the Prolog syntax to support \textbf{inline JavaScript code}, to better integrate with the browser environment.
\end{itemize}