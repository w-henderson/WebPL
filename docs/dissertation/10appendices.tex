\begin{appendices}

\chapter{JavaScript-WebAssembly Interaction}

\label{appendix:shared-memory}

To illustrate the interaction described in Section~\ref{sec:wasm-js-interface}, consider the example in Figure~\ref{fig:js-wasm} where JavaScript passes a value to WebAssembly through the linear memory, and WebAssembly calls back into JavaScript to output that value. The JavaScript \texttt{DataView} API is used to write an integer value to the linear memory, specifying little-endian byte order, and the WebAssembly function \texttt{get} loads the value and calls the imported JavaScript function \texttt{log} to output it.

\begin{figure}[H]
\centering
\begin{subfigure}{\textwidth}
\begin{minted}{javascript}
/* Instantiate the WebAssembly module, providing `console.log` as an external
 * function to call from within WebAssembly. */
WebAssembly.instantiate(fetch("a.wasm"), { env: { log: console.log } })
  .then(({ instance }) => {
    /* Use the built-in DataView API to write a value of a specific size and
     * endianness to the linear memory, in this case writing the integer 1234
     * as a 32-bit little-endian signed integer to address 0. */
    const mem = new DataView(instance.exports.memory.buffer);
    mem.setInt32(0, 1234, /*littleEndian=*/true);

    /* Call the WebAssembly function `get`, passing the address of the value
     * that was just written to the linear memory. */
    instance.exports.get(0);
  });
\end{minted}
\caption{JavaScript code to write to linear memory and call into WebAssembly}
\end{subfigure}
\par\bigskip
\par\bigskip
\begin{subfigure}{\textwidth}
\begin{minted}{asm}
(module
  ;; Import external function `log` specified when instantiating the module.
  (import "env" "log" (func $log (param i32)))

  ;; Make the linear memory accessible from JavaScript.
  (memory (export "memory") 1)

  (func (export "get") (param $ptr i32)
    local.get $ptr ;; push the pointer onto the stack
    i32.load       ;; load the value from linear memory
    call $log      ;; pass the value to the JavaScript function
  )
)
\end{minted}
\caption{WebAssembly code to read from linear memory and call into JavaScript}
\end{subfigure}
\caption{JavaScript-WebAssembly interaction}
\label{fig:js-wasm}
\end{figure}

\chapter{Built-in Predicates}

\label{appendix:predicates}

\section{Predicates}

Supported built-in predicates (Section~\ref{sec:builtins}) are shown in Table~\ref{table:predicates}.

\begin{table}
\centering
\begin{tabular}{lp{12cm}}
\hline
\textbf{Predicate} & \textbf{Description} \\
\hline
\texttt{is/2} & Unifies the first argument with the result of evaluating the second argument, or fails if they do not unify. \\
\texttt{=/2} & Unifies the first argument with the second argument, or fails if they do not unify. \\
\texttt{>/2} & Succeeds if the first argument is greater than the second argument. \\
\texttt{</2} & Succeeds if the first argument is less than the second argument. \\
\texttt{>=/2} & Succeeds if the first argument is greater than or equal to the second argument. \\
\texttt{=</2} & Succeeds if the first argument is less than or equal to the second argument. \\
\texttt{=\textbackslash=/2} & Succeeds if the first argument is not equal (arithmetic) to the second argument. \\
\texttt{=:=/2} & Succeeds if the first argument is equal (arithmetic) to the second argument. \\
\texttt{==/2} & Succeeds if the first argument is equal (term) to the second argument. \\
\texttt{delay/2} & Delays the evaluation of the second argument until the first is bound (see Section~\ref{sec:attributed-variables}). \\
\texttt{freeze/2} & Delays the evaluation of the second argument until the first is bound to a non-variable (see Section~\ref{sec:attributed-variables}). \\
\texttt{integer/1} & Succeeds if the argument is an integer. \\
\texttt{float/1} & Succeeds if the argument is a floating point number. \\
\texttt{atom/1} & Succeeds if the argument is an atom. \\
\texttt{var/1} & Succeeds if the argument is a variable. \\
\texttt{nonvar/1} & Succeeds if the argument is not a variable. \\
\texttt{compound/1} & Succeeds if the argument is a compound term. \\
\texttt{number/1} & Succeeds if the argument is a number (integer or float). \\
\texttt{!/0} & Cut operator. \\
\texttt{statistics/2} & Queries runtime statistics, see Table~\ref{table:statistics}. \\
\hline
\end{tabular}
\caption{Built-in predicates}
\label{table:predicates}
\end{table}

\section{Statistics Predicate}

The \texttt{statistics/2} predicate takes a statistic name as its first argument, unifying the current value of that statistic with the second argument. Available statistics are shown in Table~\ref{table:statistics}.

\begin{table}
\centering
\begin{tabular}{ll}
\hline
\textbf{Statistic} & \textbf{Description} \\
\hline
\texttt{memory} & Current heap memory usage in bytes. \\
\texttt{allocated} & Current allocated heap size in bytes. \\
\texttt{gc} & Number of garbage collections performed. \\
\texttt{wasm\_memory} & Current size of the WebAssembly linear memory in bytes. \\
\hline
\end{tabular}
\caption{Supported statistics}
\label{table:statistics}
\end{table}

\chapter{JavaScript FFI Built-In Functions}

\label{appendix:js-ffi}

\begin{table}[H]
\centering
\begin{tabular}{lp{12cm}}
\hline
\textbf{Function} & \textbf{Description} \\
\hline
\texttt{unify} & Performs unification. Both terms must be valid JavaScript representations of Prolog terms (Section~\ref{sec:js-prolog-mapping}), which will be converted back to Prolog terms and allocated on the heap if necessary. Variable bindings are checked to ensure they refer to terms that exist. \\
\texttt{fetch} & Makes a synchronous HTTP request to the given URL, returning the response as a string. This is implemented in JavaScript as a wrapper around \texttt{XMLHttpRequest}, provided for convenience. \\
\texttt{compound} & Creates a compound term from a functor and some arguments. \\
\texttt{list} & Converts a JavaScript list into its linked-list Prolog representation. \\
\hline
\end{tabular}
\caption{Functions available to JavaScript code}
\label{tab:js-ffi}
\end{table}

\chapter{Benchmark Programs}

\label{appendix:benchmarks}

\begin{table}[H]
\centering
\begin{tabular}{ll}
\hline
\textbf{Benchmark} & \textbf{Description} \\
\hline
chat\_parser & Parse natural language \\
crypt & Cryptomultiplication \\
derive & Symbolic differentiation \\
divide10 & Symbolic differentiation \\
fib & Fibonacci sequence \\
log10 & Symbolic differentiation \\
mu & MU puzzle \\
nreverse & Naive list reversal \\
ops8 & Symbolic differentiation \\
poly\_10 & Polynomial exponentiation \\
qsort & Quicksort \\
queens\_8 & 8-Queens problem \\
query & Query deductive database \\
tak & Takeuchi function \\
times10 & Symbolic differentiation \\
zebra & Zebra puzzle \\
\hline
\end{tabular}
\caption{Selected benchmarks from the SWI-Prolog benchmark suite}
\label{tab:benchmarks}
\end{table}

\chapter{Project Proposal}

\label{appendix:proposal}

\includepdf[pages=-]{../proposal/proposal.pdf}

\end{appendices}