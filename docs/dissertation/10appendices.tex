\begin{appendices}

\chapter{Project Proposal}

\label{appendix:proposal}

\includepdf[pages=-]{../proposal/proposal.pdf}

\chapter{Built-in Predicates}

\label{appendix:predicates}

\section{Predicates}

Supported built-in predicates (Section \ref{sec:builtins}) are shown in Table \ref{table:predicates}.

\begin{table}
\centering
\begin{tabular}{lp{12cm}}
\hline
\textbf{Predicate} & \textbf{Description} \\
\hline
\texttt{is/2} & Unifies the first argument with the result of evaluating the second argument, or fails if they do not unify. \\
\texttt{=/2} & Unifies the first argument with the second argument, or fails if they do not unify. \\
\texttt{>/2} & Succeeds if the first argument is greater than the second argument. \\
\texttt{</2} & Succeeds if the first argument is less than the second argument. \\
\texttt{>=/2} & Succeeds if the first argument is greater than or equal to the second argument. \\
\texttt{=</2} & Succeeds if the first argument is less than or equal to the second argument. \\
\texttt{=\textbackslash=/2} & Succeeds if the first argument is not equal (arithmetic) to the second argument. \\
\texttt{=:=/2} & Succeeds if the first argument is equal (arithmetic) to the second argument. \\
\texttt{==/2} & Succeeds if the first argument is equal (term) to the second argument. \\
\texttt{delay/2} & Delays the evaluation of the second argument until the first is bound (see Section \ref{sec:attributed-variables}). \\
\texttt{freeze/2} & Delays the evaluation of the second argument until the first is bound to a non-variable (see Section \ref{sec:attributed-variables}). \\
\texttt{integer/1} & Succeeds if the argument is an integer. \\
\texttt{float/1} & Succeeds if the argument is a floating point number. \\
\texttt{atom/1} & Succeeds if the argument is an atom. \\
\texttt{var/1} & Succeeds if the argument is a variable. \\
\texttt{nonvar/1} & Succeeds if the argument is not a variable. \\
\texttt{compound/1} & Succeeds if the argument is a compound term. \\
\texttt{number/1} & Succeeds if the argument is a number (integer or float). \\
\texttt{!/0} & Cut operator. \\
\texttt{statistics/2} & Queries runtime statistics, see Table \ref{table:statistics}. \\
\hline
\end{tabular}
\caption{Built-in predicates}
\label{table:predicates}
\end{table}

\section{Statistics Predicate}

The \texttt{statistics/2} predicate takes a statistic name as its first argument, unifying the current value of that statistic with the second argument. Available statistics are shown in Table \ref{table:statistics}.

\begin{table}
\centering
\begin{tabular}{ll}
\hline
\textbf{Statistic} & \textbf{Description} \\
\hline
\texttt{memory} & Current heap memory usage in bytes. \\
\texttt{allocated} & Current allocated heap size in bytes. \\
\texttt{gc} & Number of garbage collections performed. \\
\texttt{wasm\_memory} & Current size of the WebAssembly linear memory in bytes. \\
\hline
\end{tabular}
\caption{Supported statistics}
\label{table:statistics}
\end{table}

\end{appendices}