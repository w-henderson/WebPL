\documentclass{article}

\usepackage[a4paper, total={7in, 11in}]{geometry}

\setlength\parskip{1em}
\setlength\parindent{0pt}
\pagenumbering{gobble}

\title{Part II Project Progress Report:\\
{\bf A Prolog interpreter for the browser}}
\author{William Henderson \\
University of Cambridge \\
\texttt{wh364@cam.ac.uk}}
\date{January 2025}

\begin{document}

\maketitle

{\bf Supervisor}: Alan Mycroft \\
{\bf Director of Studies}: John Fawcett \\
{\bf Project Checkers}: Hatice Gunes, Neel Krishnaswami

\section*{Project Status}

The project, to implement a Prolog interpreter for the browser, is on schedule, with the core of the project completed, alongside a number of extensions.

\section*{Work Completed}

The success criteria have been met. This comprises a pure Prolog interpreter written in Rust (including a parser), which was compiled to WebAssembly and is able to run in the browser. I evaluated its performance against other Prolog interpreters which support the browser environment (SWI-Prolog, Trealla Prolog, and Tau Prolog) with a widely used set of benchmarks, and found its performance to be equivalent or better than the others in the majority of cases in both execution time and memory usage.

I also implemented a simple web-based IDE for the interpreter, which supports a number of other interpreters as plug-in modules (including SWI-Prolog, Trealla Prolog, and Tau Prolog). It includes syntax highlighting, benchmarking, and runs in all major browsers.

To better integrate the interpreter with the browser environment, I added support for inline JavaScript calls from Prolog via some new syntax:
\begin{center}
\texttt{add(X, Y, Z) :- <\{ (X, Y) => unify(Z, X + Y) \}>}
\end{center}

Optimisations I made include:

\begin{itemize}
\item A region-based memory allocator to exploit WebAssembly's linear memory model.
\item Choice point elimination, subsuming last call optimisation. This avoids pushing unnecessary choice points based on static analysis of the program structure.
\item Variable shunting, collapsing chains of irreversible variable bindings and reducing the size of the trail stack.
\item String interning, which reduces memory usage by storing only one copy of each unique string. I further specified fixed identifiers for built-in predicate names such as \texttt{is} to avoid string comparisons.
\item Avoiding blocking the main JavaScript thread by running WebAssembly code in a separate Web Worker.
\end{itemize}

As the browser is a memory-constrained environment, I also implemented a mark-and-sweep garbage collector for the interpreter, which improves the memory usage of long-running Prolog programs which do little backtracking.

\end{document}