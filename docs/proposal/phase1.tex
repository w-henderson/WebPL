\documentclass{article}
\usepackage[a4paper, margin=1in]{geometry}

\setlength\parskip{1em}
\setlength\parindent{0pt}

\title{Phase 1 Project Selection Status Report}
\author{William Henderson \\
University of Cambridge}
\date{September 2024}

\begin{document}

\maketitle

Subject: Phase 1 - Henderson: A Browser-Optimised Prolog Interpreter

{\bf Name:} William Henderson \\
{\bf College:} Churchill College \\
{\bf CRSID:} wh364 \\
{\bf Director of Studies:} Dr John Fawcett

\begin{enumerate}

\item {\bf Please write 100 words on your current project ideas.}

The project plan is to build an interpreter for pure Prolog, optimised for browser execution, in Rust, and compile it to the browser-compatible binary format WebAssembly. As an extension, I will build a browser-based IDE to demonstrate the interpreter’s functionality, which will also facilitate comparison with other Prolog implementations by allowing the user to choose the underlying interpreter. Existing browser-based Prolog IDEs tend to either use JavaScript, a standard Prolog interpreter like SWI-Prolog compiled to WebAssembly, or a client-server model where code is sent to a server for execution, whereas my approach optimises from first principles specifically for this use-case.

\item {\bf Please list names of potential project supervisors, indicating any interactions you have had with them, for example: not contacted, awaiting reply, in discussion, agreed to supervise.}

Alan Mycroft has agreed to supervise.

\item {\bf Is there any chance that your project will involve any computing resources other than the Computing Service's MCS and software that is already installed there, for example: your own machine, machines in College, special peripherals, imported software packages, special hardware, network access, substantial extra disc space on the MCS.

If so indicate below what, and what it is needed for.}

I will be using my own machine for development. Necessary software packages will include the Rust toolchain, LALRPOP (for parsing Prolog code (for compiling to WebAssembly), and possibly others.

\end{enumerate}

\end{document}